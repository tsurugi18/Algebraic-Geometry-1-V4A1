\documentclass{article}

\usepackage{amsmath}
\usepackage{amssymb}
\usepackage{amsthm}
\usepackage{enumerate}
\usepackage{bbm}
\usepackage{lipsum}
\usepackage{fancyhdr}
\usepackage{calrsfs}
\usepackage{tikz-cd} 

\newtheorem{theorem}{Theorem}[section] 
\newtheorem{proposition}{Proposition}[section] 
\newtheorem{definition}{Definition}[section] 
\newtheorem{lemma}{Lemma}[section] 
\newtheorem{notation}{Notation}[section] 
\newtheorem{remark}{Remark}[section] 
\newtheorem{corollary}{Corollary}[section] 
\newtheorem{terminology}{Terminology}[section] 
\newtheorem{example}{Example}[section] 
\numberwithin{equation}{section}

\DeclareMathOperator{\diam}{diam}
\DeclareMathOperator{\rk}{rk}
\DeclareMathOperator{\rank}{rank}
\DeclareMathOperator{\Hom}{Hom}
\DeclareMathOperator{\Dom}{Dom}
\DeclareMathOperator{\grad}{grad}
\DeclareMathOperator{\Span}{Span}
\DeclareMathOperator{\interior}{int}
\DeclareMathOperator{\ind}{ind}
\DeclareMathOperator{\supp}{supp}
\DeclareMathOperator{\sgn}{sgn}
\DeclareMathOperator{\ob}{ob}
\DeclareMathOperator{\Spec}{Spec}
\DeclareMathOperator{\PreSh}{PreSh}
\DeclareMathOperator{\Fun}{Fun}
\DeclareMathOperator{\Ker}{Ker}
\DeclareMathOperator{\Image}{Im}
\DeclareMathOperator{\Ad}{Ad}
\DeclareMathOperator{\ad}{ad}
\DeclareMathOperator{\End}{End}
\DeclareMathOperator{\GL}{GL}
\DeclareMathOperator{\SL}{SL}
\DeclareMathOperator{\Lie}{Lie}

\title{Sheet 7 V4A1}
\author{}
\date{}

\begin{document}
\maketitle

\section*{39}

\subsection*{(ii)}
Let $K/k$ be an arbitrary transcendental extension of fields. And $f:\Spec(K)\to\Spec(k)$ be a natural mapping (ie. mapping the unique point to the unique point). This is not of finite type. But it has finite fibers, but it has a base change with infinite fibers. In particular, take $T\in K$ to be a transcendental element over $k$, and we claim that the base change by $\Spec(k(T))\to \Spec(k)$ has infinite fibers. This base change is $\Spec(K\otimes_kk(T))\to\Spec(k(T))$, which factors as
\begin{equation*}
\Spec(K\otimes_kk(T))\to\Spec(k(T)\otimes_kk(T))\to\Spec(k(T))
\end{equation*}
whose first step is surjective (being a base change of the surjection $\Spec(K)\to\Spec(k(T))$, so it is enough that the second map $q:\Spec(k(T)\otimes_kk(T))\to\Spec(k(T))$ has infinite fiber. But we described $k(T)\otimes_kk(T)$ above and its Spec is infinite.

This shows both that quasi-finite and injectivity are not preserved under morphisms.

\subsection*{(iii)}
Let $f:Z\to X$ be a closed immersion. In other words, there is an open affine covering $\{U_i\}$, with $U_i=\Spec(A_i)$ of $X$ such that $f^{-1}(U_i)=\Spec(A_i/I_i)$ for some ideal $I_i$ of $A_i$. \\
\par Let $g:Y\to X$ be a morphism. To show that $q:Z\times_XY\to Y$ is a closed immersion,
we take an open affine covering $\{\Spec(B_i)\}$ of $Y$ such that $g(\Spec(B_i))\subset(\Spec(A_i))$. \\
\par By since $q(\Spec(B_i)) = (\Spec(B_i))\times_XZ = (\Spec(B_i))\times_{\Spec(A_i)}\Spec(A_i/I_i) = \Spec(B_i\otimes_{A_i}A_i/I_i) = \Spec(B_i/I_iB_i)$, therefore $q$ is a closed immersion.\\
\par For open immersion, we have

\section*{40}

We have that $U\times_SV$ is an open subscheme of $X\times_SX$. 

\[
\begin{tikzcd}
X \arrow[r, "\Delta", bend left=49] & X\times_SX \arrow[l, "p"] \arrow[r, "q"']                                                                    & X \arrow[l, "\Delta"', bend right=49] \\
U \arrow[u, "i_U"]                  & U\times_S V \arrow[l, "p_U"'] \arrow[r, "q_V"] \arrow[u, "i_{UV}"]                                           & V \arrow[u, "i_V"']                   \\
                                    & U\cap V \arrow[lu, "\iota_U"] \arrow[ru, "\iota_V"'] \arrow[u, "\theta_{UV}"'] \arrow[uu, dashed, bend left] &                                      
\end{tikzcd}
\]

where $i_U,i_V,\iota_U,\iota_V$ are inclusions and $\theta_{UV}$ is an isomorphism of $U\times_SV$ and $U\cap V$, this follows from that $\Delta^{-1}(U\times_SV)$.\\
\par By the universal property of $X\times_S X$, we have
\begin{equation*}
h = i_{UV}\theta_{UV}:U\cap V\to X\times_S X, \quad p\circ h  = i_U\iota_U, q\circ h = i_v\iota_V.
\end{equation*}
On the other hands, there are $\delta i_U\iota_U, \delta i_V\iota_V$ satisfying the conditions as $h$. We conclude that 
\begin{equation*}
p_1\circ(\Delta i_u\iota_U)=id_Xi_U\iota_U = i_U\iota_U, p_2\circ(\Delta i_V\iota_V) = i_V\iota_V.
\end{equation*}
Again using the universal property, we derive
\begin{equation*}
h = \Delta i_U\iota_U = i_{UV}\theta_{UV}\Rightarrow \Delta(U\cap V) = \theta_{UV}(U\cap)\subseteq U\times_S V\Rightarrow U\cap V\supseteq\Delta^{-1}(U\times_SV).
\end{equation*}
Also we have
\begin{equation*}
U\subseteq p(U\times_S V)\subseteq p_1\Delta\Delta^{-1}(U\times_S V) = \Delta^{-1}(U\times_S V),
\end{equation*}
we derive $V\subseteq\Delta^{-1}(U\times_SV)$, we conclude $U\cap V\subseteq \Delta^{-1}(U\times_SV)$.\\
\par We have $U,V,S$ are affine therefore, $U\times_S V$ is affine. Since $U\cap V=\Delta^{-1}(U\times_S V)$. $\Delta|_{U\cap V}$ is a restriction of a closed immersion, there fore a closed immersion. We conclude that $U\cap V$ is affine.
\end{document}