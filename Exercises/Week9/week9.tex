\documentclass{article}

\usepackage{amsmath}
\usepackage{amssymb}
\usepackage{amsthm}
\usepackage{ mathrsfs }
\usepackage{enumerate}
\usepackage{bbm}
\usepackage{lipsum}
\usepackage{fancyhdr}
\usepackage{tikz-cd} 

\newtheorem{theorem}{Theorem}[section] 
\newtheorem{proposition}{Proposition}[section] 
\newtheorem{definition}{Definition}[section] 
\newtheorem{lemma}{Lemma}[section] 
\newtheorem{notation}{Notation}[section] 
\newtheorem{remark}{Remark}[section] 
\newtheorem{corollary}{Corollary}[section] 
\newtheorem{terminology}{Terminology}[section] 
\newtheorem{example}{Example}[section] 
\numberwithin{equation}{section}

\DeclareMathOperator{\diam}{diam}
\DeclareMathOperator{\Gal}{Gal}
\DeclareMathOperator{\rank}{rank}
\DeclareMathOperator{\Hom}{Hom}
\DeclareMathOperator{\Dom}{Dom}
\DeclareMathOperator{\Ann}{Ann}
\DeclareMathOperator{\grad}{grad}
\DeclareMathOperator{\Span}{Span}
\DeclareMathOperator{\interior}{int}
\DeclareMathOperator{\ind}{ind}
\DeclareMathOperator{\supp}{supp}
\DeclareMathOperator{\ob}{ob}
\DeclareMathOperator{\Spec}{Spec}
\DeclareMathOperator{\MaxSpec}{MaxSpec}
\DeclareMathOperator{\PreSh}{PreSh}
\DeclareMathOperator{\Sh}{Sh}
\DeclareMathOperator{\Fun}{Fun}
\DeclareMathOperator{\Ext}{Ext}
\DeclareMathOperator{\Aut}{Aut}
\DeclareMathOperator{\Ker}{Ker}
\DeclareMathOperator{\Image}{Im}
\DeclareMathOperator{\Coker}{Coker}
\DeclareMathOperator{\id}{id}
%\newcommand*{\name}[\num_arguments][default values]{{\color{#1}\Large #2}}
\newcommand*{\multivar}[2]{{#1_1,\cdots,#1_{#2}}}
\newcommand*{\localization}[2]{#1_{\mathfrak{#2}}}
\newcommand*{\ringedspacemorph}[1]{(#1,#1^{\#})}
\newcommand*{\sheaf}[2]{(#1,{\mathcal{#2}}_{#1})}
\newcommand{\fib}[1]{%
  \mathbin{\mathop{\times}\limits_{#1}}%
}

\title{Algebraic Geometry 1}
\author{So Murata}
\date{2024/2025 Winter Semester - Uni Bonn}

\begin{document}
\maketitle
\section*{50}

By taking the global section of 
\begin{equation*}
\tilde{M}\to\mathscr{F},
\end{equation*}
it is clear that the morphism
\begin{equation*}
\alpha:\Hom_{\mathcal{O}_x}(\tilde{M},\mathscr{F})\to\Hom_A(M,\Gamma(X,\mathscr{F}))
\end{equation*}
exists. Let $\varphi:M\to\mathscr{F}(X)$ and for $X=\bigcup D(a)$, we define
\begin{equation*}
\varphi_a:\tilde{M}(D(a))=M_a\to\mathscr{F}(D(f))., {\frac m {a^i}} \mapsto {\frac {\varphi(m)|_{D(a)}} {a^i}}.
\end{equation*}
Since $\mathscr{F}(D(a))$ i an $A_a$-module and ${\frac 1 {a^i}}\in A_a$, we have 
\begin{equation*}
{\frac {\varphi(m)|_{D(a)}} {a^i}}\in\mathscr{F}(D(a)).
\end{equation*}
In particular $\varphi_a({\frac m 1}) = \varphi(m)|_{D(a)}$.\\
\par By $D(a)\cap D(b) = D(ab)$ and the sheaf property if we have
\begin{equation*}
(\varphi_a)_{ab} = (\varphi_b)_{ab},
\end{equation*}
then we can glue $(\varphi_a)$ to $\varphi:\tilde{M}\to\mathscr{F}$. 
\begin{align*}
(\varphi_a)_{ab}:(M_a)_{ab} &= M_{ab} \in{\frac m {(ab)^i)}} = {\frac {{\frac 1 m}} {(ab)^i)}}\\
&\mapsto {\frac {\varphi_a({\frac m 1})|_{D(ab)}} {(ab)^i}}\\
& = {\frac {(\varphi(m)|_{D(a)})|_{D(ab)}} {(ab)^i}}\\
& = {\frac {\varphi(m)|_{D(ab)}} {(ab)^i}}\in\mathscr{D(ab)}.
\end{align*}
The last equality follows from the composition of restrictions. Therefore
\begin{equation*}
(\varphi_a)_{ab}\left({\frac m {(ab)^i}}\right) = {\frac {\varphi(m)|_{D(ab)}} {(ab)^i}} = (\varphi_b)_{ab}\left({\frac m {(ab)^i}}\right).
\end{equation*}
Thus we can glue $(\varphi_a)$ to get $\varphi:\tilde{M}\to\mathscr{F}$. Thus there exists
\begin{equation*}
\beta:\Hom_A(M,\Gamma(X,\mathscr{F}))\to\Hom_{\mathcal{O}_X}(\tilde{M},\mathscr{F}).
\end{equation*}

By the construction of $\beta(\varphi)$, taking the global section of it we get 
\begin{equation*}
\varphi:M\to\mathscr{F}(X).
\end{equation*}

Thus we get $\alpha\beta(\varphi) = \varphi$. On the other hands, by the uniqueness of glueing process, we have 
\begin{equation*}
\beta\alpha = \id_{\Hom_{\mathcal{O}_X}(\tilde{M},\mathscr{F})}.
\end{equation*}

In conclusion, we have
\begin{equation*}
\Hom_{\mathcal{O}_X}(\tilde{M},\mathscr{F}) \cong\Hom_A(M,\Gamma(X,\mathscr{F})).
\end{equation*}

\end{document}