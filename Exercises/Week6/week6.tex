\documentclass{article}

\usepackage{amsmath}
\usepackage{amssymb}
\usepackage{amsthm}
\usepackage{enumerate}
\usepackage{bbm}
\usepackage{lipsum}
\usepackage{fancyhdr}
\usepackage{calrsfs}
\usepackage{tikz-cd} 

\newtheorem{theorem}{Theorem}[section] 
\newtheorem{proposition}{Proposition}[section] 
\newtheorem{definition}{Definition}[section] 
\newtheorem{lemma}{Lemma}[section] 
\newtheorem{notation}{Notation}[section] 
\newtheorem{remark}{Remark}[section] 
\newtheorem{corollary}{Corollary}[section] 
\newtheorem{terminology}{Terminology}[section] 
\newtheorem{example}{Example}[section] 
\numberwithin{equation}{section}

\DeclareMathOperator{\diam}{diam}
\DeclareMathOperator{\rk}{rk}
\DeclareMathOperator{\rank}{rank}
\DeclareMathOperator{\Hom}{Hom}
\DeclareMathOperator{\Dom}{Dom}
\DeclareMathOperator{\grad}{grad}
\DeclareMathOperator{\Span}{Span}
\DeclareMathOperator{\interior}{int}
\DeclareMathOperator{\ind}{ind}
\DeclareMathOperator{\supp}{supp}
\DeclareMathOperator{\sgn}{sgn}
\DeclareMathOperator{\ob}{ob}
\DeclareMathOperator{\Spec}{Spec}
\DeclareMathOperator{\PreSh}{PreSh}
\DeclareMathOperator{\Fun}{Fun}
\DeclareMathOperator{\Ker}{Ker}
\DeclareMathOperator{\Image}{Im}
\DeclareMathOperator{\Ad}{Ad}
\DeclareMathOperator{\ad}{ad}
\DeclareMathOperator{\End}{End}
\DeclareMathOperator{\GL}{GL}
\DeclareMathOperator{\SL}{SL}
\DeclareMathOperator{\Lie}{Lie}


\title{Representation Theory 1 V4A1 Sheet 6 Exercise 30,32}
\author{So Murata}
\date{2024/2025 Winter Semester - Uni Bonn}

\begin{document}
\maketitle

\section*{30}
Let $f:X\to k$ be a regular function. Since $\mathbb{P}^1_k$ is irreducible, any open sets in $X$ intersects with any other open sets. Therefore, let $x,y\in X$ and $U_x,U_y$ be open sets such that
\begin{equation*}
f|_{U_x}={\frac {p_x} {q_x}},f|_{U_y}={\frac {p_y} {q_y}}.
\end{equation*}
Then $U_x\cap U_y\not=\emptyset$, thus two rational functions agree on an open set, thus we conclude
\begin{equation*}
f={\frac p q}.
\end{equation*}
Since $k$ is algebraically closed, $q$ has a pole unless it is a constant. If the former is the case $f$ is not regular thus $q$ is a constant. Since this is a regular function on a projective line, $p$ and $q$ have the same degree. We conclude $f$ is constant.\\

\section*{32}

Since Noetherian scheme is quasi compact, there exists a open covering consisting only of distinguished open sets $\{X_{a_i}\}_{i=1,\cdots,n}$. This follows that $X$ is locally Noetherian. Let $f_i$ be a restriction of $f$ to $X_{a_i}$. We have that 
\begin{equation*}
X_a\cap X_{a_i} = X_{aa_i}.
\end{equation*}
Since the restriction of $f$ to $X_a$ is $0$, the restriction of $f$ to $X_{aa_i}$ is also $0$. Therefore for some $n_i\in\mathbb{N}$, we have
\begin{equation*}
a^{n_i}f_i = 0
\end{equation*}
in the localization map. Since we have only finitely many $a_i$, we conclude that for some $n$
\begin{equation*}
a^nf = 0
\end{equation*}
together with the sheaf property.
\subsection*{(ii)}

We again argue by the finite cover by distinguished open sets and the sheaf property. Let $h\in \Gamma(X_a,\mathcal{O}_{X_a})$. By restricting $h$ to some $X_{a_i}$, we get 
\begin{equation*}
h_i = a^{-n_i}s_i
\end{equation*}
for some $s_i\in \mathcal{O}_{X_{a_i}}(X_{a_i})$. Argue again by the finiteness, we conclude there is $n$ such that 
\begin{equation*}
a^{n}h_i\in\mathcal{O}_{X_{a_i}}(X_{a_i})
\end{equation*}
for all $i$. Each $s_i$ can be glued since for each $i,j$
\begin{equation*}
s_i-s_j = 0
\end{equation*}
in $X_{aa_ia_j}$. This means that 
\begin{equation*}
f^{m_{ij}}(s_i-s_j)=0
\end{equation*}
for sufficiently large enough $m_{ij}$ in $X_{a_ia_j}$. Again by the finiteness argument, we conclude that $\{s_i\}$ can be glued to some $t$ in the global section. And the restriction of $t$ to $X_a$ is $a^{n}h$ for some $n$.
\subsection*{(iii)}
Let $f,g\in A$ be such that the restriction of $f$ and $g$ are equal in $X_a$. Then 
\begin{equation*}
a^n(f-g) = 0.
\end{equation*}
But this means that $f$ and $g$ are the same in the localization $A_a$. On the other hand for any $h\in \mathcal{O}_{X_{a}}(X_{a})$ is a restriction of some $t\in A$. Thus $\mathcal{O}_{X_{a}}(X_{a})$ satisfies the properties of localization. By the uniqueness of localizations up to isomorphisms, we conclude the statement.
\end{document}