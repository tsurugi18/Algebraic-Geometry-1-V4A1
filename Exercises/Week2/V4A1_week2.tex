\documentclass{article}

\usepackage{amsmath}
\usepackage{amssymb}
\usepackage{amsthm}
\usepackage{enumerate}
\usepackage{bbm}
\usepackage{lipsum}
\usepackage{fancyhdr}
\usepackage{calrsfs}
\usepackage{tikz-cd} 

\newtheorem{theorem}{Theorem}[section] 
\newtheorem{proposition}{Proposition}[section] 
\newtheorem{definition}{Definition}[section] 
\newtheorem{lemma}{Lemma}[section] 
\newtheorem{notation}{Notation}[section] 
\newtheorem{remark}{Remark}[section] 
\newtheorem{corollary}{Corollary}[section] 
\newtheorem{terminology}{Terminology}[section] 
\newtheorem{example}{Example}[section] 

\DeclareMathOperator{\diam}{diam}
\DeclareMathOperator{\Ker}{Ker}
\DeclareMathOperator{\rank}{rank}
\DeclareMathOperator{\Hom}{Hom}
\DeclareMathOperator{\Dom}{Dom}
\DeclareMathOperator{\grad}{grad}
\DeclareMathOperator{\Span}{Span}
\DeclareMathOperator{\interior}{int}
\DeclareMathOperator{\ind}{ind}
\DeclareMathOperator{\supp}{supp}
\DeclareMathOperator{\ob}{ob}
\DeclareMathOperator{\Spec}{Spec}
\DeclareMathOperator{\PreSh}{PreSh}
\DeclareMathOperator{\Sh}{Sh}
\DeclareMathOperator{\Fun}{Fun}


\title{Algebraic Geometry 1 Week 2 Exercise Sheet Solutions}
\author{So Murata}
\date{2024/2025 Winter Semester - Uni Bonn}

\begin{document}
\maketitle
\section*{Exercise 8}

Let $(X,\mathcal{T})$ be a topological space and $\mathcal{F}\in\Sh_{\mathcal{A}}(X)$ be a sheaf. We define
\begin{equation*}
\vert\mathcal{F}\vert = \coprod_{x\in X}\mathcal{F}_x.
\end{equation*}
We first prove that $\mathcal{B}=\{\overline{s}(U)\:|\: U\in\mathcal{T}, \overline{s}:X\to\vert\mathcal{F}\vert\}$ defines a basis of the desired strongest topology on $\vert\mathcal{F}\vert$.
In order to do so, we need $\overline{s}^{-1}\circ\overline{s}(U)= U$ and $\mathcal{B}$ is indeed a basis of a topological space. $\overline{s}^{-1}\circ\overline{s}(U)\supseteq U$ is obvious. Therefore, we will prove the other direction of inclusion.\\
\par If $\overline{s}(x) = \overline{s}(y)$, then for any open set which contains $x$ also contains $y$. In particular, $y\in U$. Thus we have the equality $\overline{s}^{-1}\circ\overline{s}(U)= U$. And for any $s_x\in \overline{s}(U)\cap\overline{s}(V)$, then $s_x\in\overline{s}(U\cap V)$. Therefore $\mathcal{B}$ is a basis. We denote such topology as $\mathcal{T}_M$.\\
\par We now show that there is an isomorphism between the sheaf of continuous functions $f:U\to\vert\mathcal{F}\vert$ and $\mathcal{F}^+$.\\
\par First for any $s\in\mathcal{F}(U)$, $x\mapsto s_x$ defines a continuous map on the topology $\mathcal{T}_M$. Let $V\in\mathcal{T}_M$ then $V$ is of the form,
\begin{equation*}
V = \bigcup_{\lambda\in\Lambda}\overline{s}_\lambda(U_\lambda),
\end{equation*}
where $\overline{s}_\lambda$ is a map induced by $s\in\mathcal{F}(U_\lambda)$. Thus it is enough to check for some map $\overline{t}:V\to\vert\mathcal{F}\vert$, $\overline{s}^{-1}(\overline{t}(V))$ is open. \\
\par Indeed let $W = \{x\:|\: x,y\in U\cap V, s_x=t_y\}$, then this is an open map. This follows that take $W_x$ to be an open set such that $x\in W_x, s|_{W_x}=t|_{W_x}$. This is justified by the construction of stalks and germs in abelian groups. Then
\begin{equation*}
W = \bigcup_{x\in W}W_x.
\end{equation*}
And this $W$ is exactly equal to $\overline{s}^{-1}(t(U))$. \\
\par On the other hand, we prove that for any continuous section $f:U\to\vert\mathcal{F}\vert$, there is $s\in\mathcal{F}(U)$ such that $f(x) = s_x$. Take $(t,V)$ to be such that $t\in\mathcal{F}(V)$, $x\in V, t_x=f(x)$. Then $V_x=f^{-1}(t(V))$ is an open set. This means for any $y\in V_x$, $f(y)=t_y, y\in V$. Since $(V_x)_{x\in U}$ is an open covering and every pair of  terms $((t_y)_{y\in U_x})_{x\in U}$ coincide on the intersection of its domains, we can glue this to some $(s_x)_{x\in U}$. \\

\par For each $(s_x)_{x\in U},(t_x)_{x\in U}\in\mathcal{F}^+(U)$, 
\begin{equation*}
\overline{s+t}(x) = (s+t)_x = s_x+t_x, 
\end{equation*}
since restriction maps are group homomorphisms. This shows that $\mathcal{F}^+(U)\ni s\mapsto \overline{s}$ is a group homomorphism which has an inverse. Thus we have proven the statement.

\end{document}