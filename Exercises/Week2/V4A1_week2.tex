\documentclass{article}

\usepackage{amsmath}
\usepackage{amssymb}
\usepackage{amsthm}
\usepackage{enumerate}
\usepackage{bbm}
\usepackage{lipsum}
\usepackage{fancyhdr}
\usepackage{calrsfs}
\usepackage{tikz-cd} 

\newtheorem{theorem}{Theorem}[section] 
\newtheorem{proposition}{Proposition}[section] 
\newtheorem{definition}{Definition}[section] 
\newtheorem{lemma}{Lemma}[section] 
\newtheorem{notation}{Notation}[section] 
\newtheorem{remark}{Remark}[section] 
\newtheorem{corollary}{Corollary}[section] 
\newtheorem{terminology}{Terminology}[section] 
\newtheorem{claim}{Claim}

\DeclareMathOperator{\diam}{diam}
\DeclareMathOperator{\Ker}{Ker}
\DeclareMathOperator{\rank}{rank}
\DeclareMathOperator{\Hom}{Hom}
\DeclareMathOperator{\Dom}{Dom}
\DeclareMathOperator{\grad}{grad}
\DeclareMathOperator{\Span}{Span}
\DeclareMathOperator{\interior}{int}
\DeclareMathOperator{\ind}{ind}
\DeclareMathOperator{\supp}{supp}
\DeclareMathOperator{\ob}{ob}
\DeclareMathOperator{\Spec}{Spec}
\DeclareMathOperator{\PreSh}{PreSh}
\DeclareMathOperator{\Sh}{Sh}
\DeclareMathOperator{\Fun}{Fun}


\title{Algebraic Geometry 1 Week 2 Exercise Sheet Solutions}
\author{So Murata}
\date{2024/2025 Winter Semester - Uni Bonn}

\begin{document}
\maketitle

\section*{Exercise 7}
\begin{equation*}
f\mapsto (f(x))_{x\in U}
\end{equation*}
This is the sheafification of the presheaf since by the construction of germs in abelian groups, we have for $s\in\mathcal{F}(U)$,
\begin{equation*}
s_x = \{(t,V)\:|\: x\in V, \text{There exists } x\in W \text{ such that }, W\subset U,V, \rho_{UW}(s) = \rho_{VW}(t)\}.
\end{equation*}
Since $W$ is open we can take an arbitrary small open ball $B(x,{\frac 1 n})$ and any functions in $s_x$ as an equivalence class must coincide in the ball. As $n$ is arbitrary, we conclude that $f_x = f(x)$.

\section*{Exercise 8}

Let $(X,\mathcal{T})$ be a topological space and $\mathcal{F}\in\Sh_{\mathcal{A}}(X)$ be a sheaf. We define
\begin{equation*}
\vert\mathcal{F}\vert = \coprod_{x\in X}\mathcal{F}_x.
\end{equation*}
We first prove that $\mathcal{B}=\{\overline{s}(U)\:|\: U\in\mathcal{T}, \overline{s}:X\to\vert\mathcal{F}\vert\}$ defines a basis of the desired strongest topology on $\vert\mathcal{F}\vert$.
In order to do so, we will prove the following claims.
\begin{claim}
For any $s\in\mathcal{F}(U)$ and open subset $V$ of $U$, we have
\begin{equation*}
\overline{s}^{-1}\circ\overline{s}(V)= V.
\end{equation*}
\label{claim_1}
\end{claim} 
\begin{proof}
 $\overline{s}^{-1}\circ\overline{s}(V)\supseteq V$ is obvious. Therefore, we will prove the other direction of inclusion.\\
\par Let $y\in\overline{s}^{-1}\circ\overline{s}(V)$. Then $\overline{s}(y)=\overline{s}(x)$ for some $x\in V$ as an equivalence class of a pair of a section and an open sets. If $\overline{s}(x) = \overline{s}(y)$, then for any open set which contains $x$ also contains $y$. In particular, $y\in V$. Thus we have the equality 
\end{proof}
\begin{claim}
$\mathcal{B}=\{\overline{s}(U)\:|\: U\in\mathcal{T}, \overline{s}:X\to\vert\mathcal{F}\vert\}$ is a basis.
\end{claim}
\begin{proof}
For any $s_x\in \overline{s}(U)\cap\overline{t}(V)$, then $(s,U),(t,V)\in s_x$. Therefore there is $W\in U\cap V$ open such that $\rho_{UW}(s)=\rho_{VW}(t)$. In particular $s_x\in\overline{s}(W)\in\mathcal{B}$. Therefore, $\mathcal{B}$ is a basis. 
\end{proof}

We denote the topology generated by $\mathcal{B}$ as $\mathcal{T}_M$.\\
\par We now show that there is an isomorphism between the sheaf of continuous sections $f:U\to\vert\mathcal{F}\vert$ and $\mathcal{F}^+$.\\
\par First for any $s\in\mathcal{F}(U)$, $x\mapsto s_x$ defines a continuous map on the topology $\mathcal{T}_M$. Since we have a basis, it is enough to check that for each $\overline{t}:V\to\vert\mathcal{F}\vert$, $\overline{s}^{-1}(\overline{t}(V))$ is open. \\
\par Indeed let $W = \{x\:|\: x,y\in U\cap V, s_x=t_y\}$, then this is an open map. Since for each $x\in W$, we can take an open set $W_x$ such that $x\in W_x, s|_{W_x}=t|_{W_x}$. Then
\begin{equation*}
W = \bigcup_{x\in W}W_x.
\end{equation*}
And this $W$ is exactly equal to 
\begin{equation*}
\overline{s}^{-1}(t(V))=\bigcup_{x\in W}\overline{s}^{-1}(\overline{s}(W_x)) = \bigcup_{x\in W}W_x=W. 
\end{equation*}
by Claim. \ref{claim_1}.
\par On the other hand, we must prove that for any continuous section $f:U\to\vert\mathcal{F}\vert$, there is $s\in\mathcal{F}(U)$ such that $f(x) = s_x$. Take $(t,V)$ to be such that $t\in\mathcal{F}(V)$, $x\in V, t_x=f(x)$. Then $V_x=f^{-1}(t(V))$ is an open set. This means for any $y\in V_x$, $f(y)=t_y, y\in V$. Since $(V_x)_{x\in U}$ is an open covering of $U$ and every pair of  terms $((t_y)_{y\in U_x})_{x\in U}$ coincide on the intersection of its domains, we can glue this to some $s=(s_x)_{x\in U}$ and $f$ coincides with the section induced by $s$. Thus we have proven that there is a one-to-one correspondence between 
\begin{equation*}
\{\text{Sections of }\mathcal{F}^+\}\leftrightarrow\{\text{Continuous sections }f:U\to\vert\mathcal{F}\vert\}.
\end{equation*}\\

\par Suppose for $U\in\vert\mathcal{F}\vert$, we have $\overline{s}^{-1}(U)$ is open for any $s$, then for any $s_x\in U$, there is $(s^x,U_x)$, such that $(s^x,U_x)\in s_x,x\in U$. Therefore 
\begin{equation*}
U^x\cap \overline{s}^{-1}(U)
\end{equation*}
is open and 
\begin{equation*}
U=\bigcup_{s_x\in U}s(U^x\cap \overline{s}^{-1}(U)).
\end{equation*}
Therefore $U$ is contained in the topology $\mathcal{T}_M$. This shows that $\mathcal{T}_M$ is the strongest topology among all topology where all $\overline{s}$ is continuous.

\par For each $(s_x)_{x\in U},(t_x)_{x\in U}\in\mathcal{F}^+(U)$, 
\begin{equation*}
\overline{s+t}(x) = (s+t)_x = s_x+t_x, 
\end{equation*}
by definition. This shows that $\mathcal{F}^+(U)\ni s\mapsto \overline{s}$ is a group homomorphism which has an inverse. Thus we have proven the statement.

\section*{Exercise 9}

\subsection*{(i)}

Let $s,t\in\mathcal{F}(U)$. Then we let $W=\{x\in U\:|\: s_x=t_x\}$. By the construction, there exists $U_s,U_t\subset U$ open such that $x\in U_s,U_t$ and there is $x\in W_x\in U_s,U_t$ such that 
\begin{equation*}
s|_{W_x} = t|_{W_x}.
\end{equation*}

Since they coincide on $W$, we have $s_y=t_y$ for any $y\in W_x$, therefore $W_x\subset W$. Furthermore, this $W_x$ can be defined for each $x\in W$. We obtain
\begin{equation*}
W=\bigcup_{x\in W}W_x,
\end{equation*}
which is an arbitrary union of open sets, thus open.


\end{document}