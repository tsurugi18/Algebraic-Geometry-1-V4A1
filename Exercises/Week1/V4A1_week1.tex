\documentclass{article}

\usepackage{amsmath}
\usepackage{amssymb}
\usepackage{amsthm}
\usepackage{enumerate}
\usepackage{bbm}
\usepackage{lipsum}
\usepackage{fancyhdr}
\usepackage{calrsfs}
\usepackage{tikz-cd} 

\newtheorem{theorem}{Theorem}[section] 
\newtheorem{proposition}{Proposition}[section] 
\newtheorem{definition}{Definition}[section] 
\newtheorem{lemma}{Lemma}[section] 
\newtheorem{notation}{Notation}[section] 
\newtheorem{remark}{Remark}[section] 
\newtheorem{corollary}{Corollary}[section] 
\newtheorem{terminology}{Terminology}[section] 
\newtheorem{example}{Example}[section] 

\DeclareMathOperator{\diam}{diam}
\DeclareMathOperator{\Ker}{Ker}
\DeclareMathOperator{\rank}{rank}
\DeclareMathOperator{\Hom}{Hom}
\DeclareMathOperator{\Dom}{Dom}
\DeclareMathOperator{\grad}{grad}
\DeclareMathOperator{\Span}{Span}
\DeclareMathOperator{\interior}{int}
\DeclareMathOperator{\ind}{ind}
\DeclareMathOperator{\supp}{supp}
\DeclareMathOperator{\ob}{ob}
\DeclareMathOperator{\Spec}{Spec}
\DeclareMathOperator{\PreSh}{PreSh}
\DeclareMathOperator{\Fun}{Fun}


\title{Algebraic Geometry 1 Week 1 Exercise Sheet Solutions}
\author{So Murata}
\date{2024/2025 Winter Semester - Uni Bonn}

\begin{document}
\maketitle

\section*{Exercise 1}

\subsection*{(i)}
For any $\psi:P\to M$ such that $\varphi_1\circ\psi = \varphi_2\circ\psi$, thus the image of $P$ under $\psi$ is contained in $K$. Therefore, we can define a mapping $\tilde{\psi} = x\mapsto \psi(x):P\to K$. As $\varphi$ is an inclusion, $\varphi\circ\tilde{\psi}=\psi$. The uniqueness follows from the fact that $\varphi$ is inclusion and any $\psi_1,\psi_2$ such that $\varphi\circ\psi_1 = \varphi\circ\psi_2$ we have $\psi_1 = \psi_2$.

\subsection*{(ii)}
$x\in\Ker(\varphi_1-\varphi_2) \Leftrightarrow \varphi_1(x)-\varphi_2(x) = 0\Leftrightarrow \varphi_1(x)=\varphi_2(x)$. Therefore, we can use the previous argument and conclude that the inclusion is an equalizer.

\section*{Exercise 2}

\subsection{(i)}

Let $f\in\mathcal{C}(U)$. Then for any $x\in W$, we have $x\in V$ and thus $x\in U$. Therefore, $(f|_V)|_W(x) = f|_V(x) = f(x) = f|_W(x)$. In the case $W=\emptyset$ we have $\rho_UW(f) = *$, and $\rho_{VW}\circ\rho_{UV}(f) = *$ for any $f\in\mathcal{C}(U)$.

\subsection{(ii)}

Let $\psi:\mathcal{C}(U')\to \prod\mathcal{C}(V_i)$ be such that $\varphi_1\circ\psi=\varphi_2\circ\psi$ where $U\subset U'$. Let us define $\tilde{\psi}:\mathcal{C}(U')\to\mathcal{C}(U)$ to be such that 
\begin{equation*}
\tilde{\psi}(f)|_{V_i} = \psi(f)_i.
\end{equation*}

Then 
\begin{equation*}
\varphi\circ\tilde{\psi}(f)|_{V_i} = \psi(f)_i.
\end{equation*}
which is equal to $\psi$. And this $\tilde{\psi}$ is unique as $\varphi$ is injective. The injectivity comes from that $(V_i)_i$ is a covering of $U$, thus for any function $f\in\mathcal{C}(U')$ and any point $x$ has an index $i$ such that $x\in V_i$. Therefore, any image of morphism $\tilde{\psi}':\mathcal{C}(U')\to\mathcal{C}(U)$, $\tilde{\psi}(f)\in\mathcal{C}(U)$ equals to $\tilde{\psi}(f)$.


\section*{Exercise 3}

For $(m_x)_{x\in U}\in\mathcal{F}(U)$, we define the map $\rho_{UV}((m_x)_{x\in U}) = (m_y)_{y\in V}$. i) follows from the inclusion relations for $W\subset V\subset U$. 

\subsection*{(ii)}



\section*{Exercise 4}

\subsection*{Construction}
Let us define a presheaf such that
\begin{equation*}
\mathcal{O}(U) = \bigcap_{(p)\in U} \mathbb{Z}_{(p)}.
\end{equation*}

Then this is isomorphic to 
\begin{equation*}
\mathbb{Z}\left[{\frac 1 n}\right]\quad\text{where }n=\prod_{i=1}^n p_i,
\end{equation*}
such that each $\{p_i\}_{i=1,\cdots,n}$ is the set of all primes not contained in any of $(p)\in U$.\\
We can assure that there are finitely many such primes. By the definition, the closed set of the Zariski topology is of the form $V(\mathfrak{a})$ for some ideal $\mathfrak{a}\subset\mathbb{Z}$. $\mathbb{Z}$ is a principle ideal domain, there is $a\in\mathbb{Z}$ such that $\mathfrak{a} = (a)$. Since $a$ has finitely many prime divisors, we now conclude that there are finitely many primes belong to $X\backslash U$.\\
\par We will now show that $\mathcal{O}(U)=\mathbb{Z}\left[{\frac 1 n}\right]$. Any ${\frac a b}\in\mathcal{O}(U)$, $b\in\mathbb{Z}\backslash (p)$ for each $(p)\in U$. Therefore, the primes that appear in the prime decomposition of $b$ is the subset of  $\{p_i\}_{i=1,\cdots,n}$. As ${\frac 1 {p_i}} ={\frac {\prod_{j\not=i} p_j} {n}}$, ${\frac 1 {p_i}}\in\mathbb{Z}[{\frac 1 n}]$. This implies that $\bigcap_{(p)\in U} \mathbb{Z}_{(p)}\subseteq \mathbb{Z}[{\frac 1 n}]$. By the construction of $n$, we have $\mathbb{Z}[{\frac 1 n}]\subseteq \bigcap_{(p)\in U} \mathbb{Z}_{(p)}$. Hence they are isomorphic.\\

\subsection*{(i)}

\par Let us now define $\rho_{UV}({\frac a b}) = {\frac a b}$, then this is well-defined when $V\subseteq U$ since such inclusion implies $\bigcap_{(p)\in U} \mathbb{Z}_{(p)}\subseteq \bigcap_{(q)\in V} \mathbb{Z}_{(q)}$.\\

\par Thus for any $W\subset V\subset U\subset X$ open,
\begin{equation*}
\rho_{UW}\left({\frac a b}\right) = {\frac a b} = \rho_{VW}\left({\frac a b}\right) = \rho_{VW}\left(\rho_{UV}\left({\frac a b}\right)\right).
\end{equation*}
Therefore, $\rho_{UW}=\rho_{VW}\circ\rho_{UV}$.\\
\par As in the previous exercise, for an open covering $(V_i)_{i\in I}$ with $U=\bigcup_{i\in I}V_i$, we define $\varphi_1,\varphi_2:\prod_{i\in I}\mathcal{O}(V_i)\to\prod_{i,j\in I}\mathcal{O}(V_i\cap V_j)$ such that
\begin{equation*}
\varphi_1\left({\frac {a_i} {b_i}}\right) = \left(\rho_{UV_i\cap V_j}\left({\frac {a_i} {b_i}}\right)\right)_{i,j\in I}, \varphi_2\left({\frac {a_i} {b_i}}\right) = \left(\rho_{UV_i\cap V_j}\left({\frac {a_j} {b_j}}\right)\right)_{i,j\in I}.
\end{equation*}

\subsection*{(ii)}

We will show that 
\begin{equation*}
\varphi:\mathcal{O}(U)\to\prod_{i\in I} \mathcal{O}(V_i), \quad\varphi\left({\frac a b}\right) = \left(\rho_{UV_i}\left({\frac a b}\right)\right)
\end{equation*}

is an equalizer of $(\varphi_1,\varphi_2)$.\\
\par Suppose for $ U'\subset X$ open and there is $\psi:\mathcal{O}(U')\to\prod_{i\in I}\mathcal{O}(V_i)$ such that 
\begin{equation}
\label{eq_cond_3_2}
\varphi_1\circ\psi = \varphi_2\circ\psi.
\end{equation}
We define $\tilde{\psi}\left({\frac a b}\right)=\psi\left({\frac a b}\right)_1$. By Equation \ref{eq_cond_3_2}, we get that for each $\psi\left({\frac a b}\right)_i=\psi\left({\frac a b}\right)_j$ in $V_i\cap V_j$ for any $i,j\in I$. Hence for each $i\in I$,
\begin{equation}
\varphi(\tilde{\psi})\left({\frac a b}\right)_i = \psi\left({\frac a b}\right)_1 =  \psi\left({\frac a b}\right)_i.
\end{equation}
We've proven that $\varphi\circ\tilde{\psi} = \psi$.\\
\par Conversely, suppose $\tilde{\psi}$ is such that $\varphi\circ\tilde{\psi} = \psi$. Then for any $i\in I$.
\begin{equation*}
\psi\left({\frac a b}\right)_i=\rho_{UV_i}\left(\tilde{\psi}\left({\frac a b}\right)\right)=\tilde{\psi}\left({\frac a b}\right).
\end{equation*}
In particular,
\begin{equation*}
\psi\left({\frac a b}\right)_1=\tilde{\psi}\left({\frac a b}\right).
\end{equation*}
\section{Exercise 5}

\par Let $\mathcal{F},\mathcal{G}$ be presheaves on a topological space $X$. We need to find homomorphisms $\rho_{UV}:\Hom(\mathcal{F}_U,\mathcal{G}_U)\to\Hom(\mathcal{F}_V,\mathcal{G}_V)$ such that $\rho_{VW}\circ\rho_{UV}=\rho_{UW}$ holds
for any $W\subset V\subset U\subset X$ open. Let $\alpha:\mathcal{F}_U\to\mathcal{G}_U$ be a presheaf homomorphism. We define $\rho_{UV}(\alpha)$ to be such that for any $W\subset V$, $\rho_{UV}(\alpha)_W = \alpha_W$. By this construction, $\rho_{UU}=\mathbf{id}_{\Hom(\mathcal{F}_U,\mathcal{G}_U)}$. And this satisfies the inclusion criterion, thus a presheaf. 
\\
\par Let us denote the presheaf by $\mathcal{H}(U) = \Hom(\mathcal{F}|_U,\mathcal{G}|_U)$. Suppose $\{U_i\}_{i\in I}$ is an open covering of $U$. Then we define $\varphi,\varphi_1,\varphi_2$ as in the definition of sheaves. We will now show that $\varphi$ is an equalizer of $(\varphi_1,\varphi_2)$. \\
\par Suppose there is $\psi:\mathcal{H}(U')\to\prod_{i\in I}\mathcal{H}(U_i)$ such that $\varphi_1\circ\psi = \varphi_2\circ\psi$.  Then we will define $\tilde{\psi}$ to be
\begin{equation*}
\tilde{\psi}(\alpha)(V) = \alpha(V).
\end{equation*}



\section{Exercise 6}

Let us define a collection of open sets $U_{(x,n)} = B(x,{\frac {|x|} n})$. Then $\mathcal{B}=\{U_{(x,n)}\}_{X\times\mathbb{N}}$ forms a basis of the space which each basis element is simply connected as they do not contain the origin.\\

\par In order to show the subjectivity, we first note that by Cauchy's integral formula, we have that for any holomorphic function $f:U\to\mathbb{C}$ over a simply connected domain $U$, there exists a anti-derivative $F:U\to\mathbb{C}$ of $f$ on $U$. \\
\par For $U\in\mathcal{B}$ and $f\in\mathcal{O}(U)^*_X$, let us define a function $h:\mathcal{U}\to\mathbb{C}$, such that
\begin{equation*}
h(z) = {\frac {f'(z)} {f(z)}}.
\end{equation*}

Since $f$ is nowhere $0$, this is a well-defined holomorphic function on a simply connected open set $U$. Thus it has an anti-derivative $H:U\to\mathbb{C}$.\\
\par We know show that $f=\exp(H)$. 

\begin{align*}
(f\exp(-H))' & = f'\exp(-H)+f(-H')\exp(-H),\\
& = f'\exp(-H)-f'\exp(-H),\\
& = 0.
\end{align*}
From this we conclude that $f(z) = c\exp(H)=\exp(H+C)$, for some constants $c,C\in\mathbb{C}$. Thus the map $\exp:\mathcal{O}_X(U)\to\mathcal{O}^*_X(U)$ is a surjection.

We now examine the kernel of the morphism. Let $z\in U$,
\begin{equation*}
e^{f(z)} = e^{g(z)} \Rightarrow {\frac {e^{f(z)}} {e^{g(z)}}} = 1 = e^{f(z)-g(z)} = e^{2\pi ki}
\end{equation*}
for some $k\in\mathbb{Z}$. By the continuity of $f,g$, we have that $f-g$ is also continuous and the values $f-g$ can take is the discrete set $2\pi i\mathbb{Z}$ . Therefore $f-g$ is a fixed value $2\pi ki$ for some $k\in\mathbb{Z}$. We conclude $\Ker(\exp)=2\pi i\mathbb{Z}$.

\end{document}