\documentclass{article}

\usepackage{amsmath}
\usepackage{amssymb}
\usepackage{amsthm}
\usepackage{enumerate}
\usepackage{bbm}
\usepackage{lipsum}
\usepackage{fancyhdr}
\usepackage{calrsfs}
\usepackage{tikz-cd} 

\newtheorem{theorem}{Theorem}[section] 
\newtheorem{proposition}{Proposition}[section] 
\newtheorem{definition}{Definition}[section] 
\newtheorem{lemma}{Lemma}[section] 
\newtheorem{notation}{Notation}[section] 
\newtheorem{remark}{Remark}[section] 
\newtheorem{corollary}{Corollary}[section] 
\newtheorem{terminology}{Terminology}[section] 
\newtheorem{example}{Example}[section] 
\numberwithin{equation}{section}

\DeclareMathOperator{\diam}{diam}
\DeclareMathOperator{\rank}{rank}
\DeclareMathOperator{\Hom}{Hom}
\DeclareMathOperator{\Dom}{Dom}
\DeclareMathOperator{\grad}{grad}
\DeclareMathOperator{\Span}{Span}
\DeclareMathOperator{\interior}{int}
\DeclareMathOperator{\ind}{ind}
\DeclareMathOperator{\supp}{supp}
\DeclareMathOperator{\ob}{ob}
\DeclareMathOperator{\Spec}{Spec}
\DeclareMathOperator{\PreSh}{PreSh}
\DeclareMathOperator{\Sh}{Sh}
\DeclareMathOperator{\Fun}{Fun}
\DeclareMathOperator{\Ker}{Ker}
\DeclareMathOperator{\Image}{Im}
\DeclareMathOperator{\Coker}{Coker}

\title{Algebraic Geometry 1 Sheet 4 Problem 22}
\author{So Murata}
\date{2024/2025 Winter Semester - Uni Bonn}

\begin{document}
\maketitle

\section*{Exercise 22}

Let us pick a short exact sequence of sheaves.
\begin{equation*}
0\to\mathcal{G}_1\stackrel{i}{\to}\mathcal{G}_2\stackrel{\pi}{\to}\mathcal{G}_3\to0.
\end{equation*}

Then we claim that 
\begin{equation*}
0\to\Hom(\mathcal{F},\mathcal{G}_1)\stackrel{i}{\to}\Hom(\mathcal{F},\mathcal{G}_2)\stackrel{\pi}{\to}\Hom(\mathcal{F},\mathcal{G}_3),
\end{equation*}
is an exact sequence. Indeed for any $\varphi,\psi\in\Hom(\mathcal{F},\mathcal{G}_1)$, we have 
\begin{equation*}
i\circ\varphi=i\circ\psi
\end{equation*}
implies, $\varphi|_U=\psi|_U$ for any open set $U\subseteq X$ by the injectivity of $i$. Furthermore, for any $\varphi\in\Ker\pi$, we have $\Image\varphi|_U\subseteq\Image i_U$. Thus $\varphi |_U$ factors through $\Image i_U$, there is $\tilde{\varphi}|_U:\mathcal{F}(U)\to\mathcal{G}_1(U)$ such that $i|_U\circ\tilde{\varphi}|_U=\varphi|_U$. The other way is trivial as $\pi\circ i\circ\varphi=0$ by the definition of sheaf morphisms. Thus we have proven the exactness at $\mathcal{G}_2$.\\
\par Now we show that $\Hom(\cdot,\mathcal{G})$ is a left exact covariant functor. Indeed, we are given
\begin{equation*}
0\to\mathcal{F}_1\stackrel{i}{\to}\mathcal{F}_2\stackrel{\pi}{\to}\mathcal{F}_3\to0,
\end{equation*}
we will show that 
\begin{equation*}
0\to\Hom(\mathcal{F}_3,\mathcal{G})\stackrel{\pi}{\to}\Hom(\mathcal{F}_2,\mathcal{G})\stackrel{i}{\to}\Hom(\mathcal{F}_1,\mathcal{G})
\end{equation*}
is exact. By the definition of epimorphism, we have for any $\varphi,\psi\in\Hom(\mathcal{F}_3,\mathcal{G})$
\begin{equation*}
\psi\circ\pi = \varphi\circ\pi
\end{equation*}
then $\psi=\varphi$. Furthermore, if $\varphi\circ i = 0$ then for any $U\subseteq X$ open we have
\begin{equation*}
\Image i|_U\subseteq \Ker\varphi|_U.
\end{equation*}
And $\Image i|_U=\Ker\pi|_U$ by the exactness. Therefore, $\varphi$ factors through $\Ker\pi|_U$, there is $\psi$ such that $\psi\pi=\varphi$.
\end{document}