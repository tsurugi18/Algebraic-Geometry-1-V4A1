\documentclass{article}

\usepackage{amsmath}
\usepackage{amssymb}
\usepackage{amsthm}
\usepackage{enumerate}
\usepackage{bbm}
\usepackage{lipsum}
\usepackage{fancyhdr}
\usepackage{calrsfs}
\usepackage{tikz-cd} 

\newtheorem{theorem}{Theorem}[section] 
\newtheorem{proposition}{Proposition}[section] 
\newtheorem{definition}{Definition}[section] 
\newtheorem{lemma}{Lemma}[section] 
\newtheorem{notation}{Notation}[section] 
\newtheorem{remark}{Remark}[section] 
\newtheorem{corollary}{Corollary}[section] 
\newtheorem{terminology}{Terminology}[section] 
\newtheorem{claim}{Claim}

\DeclareMathOperator{\diam}{diam}
\DeclareMathOperator{\Ker}{Ker}
\DeclareMathOperator{\Coker}{Coker}
\DeclareMathOperator{\rank}{rank}
\DeclareMathOperator{\Hom}{Hom}
\DeclareMathOperator{\Dom}{Dom}
\DeclareMathOperator{\grad}{grad}
\DeclareMathOperator{\Span}{Span}
\DeclareMathOperator{\interior}{int}
\DeclareMathOperator{\ind}{ind}
\DeclareMathOperator{\supp}{supp}
\DeclareMathOperator{\ob}{ob}
\DeclareMathOperator{\Spec}{Spec}
\DeclareMathOperator{\PreSh}{PreSh}
\DeclareMathOperator{\Sh}{Sh}
\DeclareMathOperator{\Fun}{Fun}


\title{Algebraic Geometry 1 Week 3 Exercise Sheet Solutions}
\author{So Murata}
\date{2024/2025 Winter Semester - Uni Bonn}

\begin{document}
\maketitle

\section*{Exercise 15}

Let us consider a short exact sequence
\begin{equation*}
0\to\underline{Z}\overset{2\pi}{\to}\mathcal{O}_{S^1}\overset{\exp}{\to}\mathcal{O}^*_{S^1}\to1.
\end{equation*}

Since each $s:S^1\to\mathbb{Z}$ is continuous, $\bigcup_{n}\in\mathbb{Z}s^{-1}(n)$ is an open cover of $S^1$ and each of them are disjoint. By the connectedness of $S^1$ we conclude that $s$ is constant on $S^1$. In general, $s\in\underline{Z}(U)$ is constant on each connected component in $U$. From the last exercise of the first week exercise sheet we know that $\Coker f_*\exp=0$, therefore by the definition of the exact sequence we conclude that $R^1f_*\underline{Z}=0$.  

For any connected open set $U$ in $S^1$, $f^{-1}(U) = \left({\frac 1 2}U\right)\cup\left({\frac 1 2}U+\pi\right)$. Thus it consists of two disjoint connected components. When taking direct limits, we can restrict the indexing open sets to be connected. Similarly from the 6th exercise, we conclude that 
\begin{equation*}
(f_*\underline{Z})_x = (2\pi i\mathbb{Z})^{\pi_0(\left({\frac 1 2}U\right)\cup\left({\frac 1 2}U+\pi\right))} \cong \mathbb{Z}\oplus\mathbb{Z}.
\end{equation*}
\section*{Exercise 16}

We know that $\Sh(\{x\})=(\mathbf{Ab})$. And for such inclusion $i:\{x\}\to X$, 
\begin{equation*}
i_{x*}(G)(U) = 
\begin{cases}
G\quad (x\in U),\\
\emptyset \quad(x\not\in U).
\end{cases}
\end{equation*}

Given an exact sequence of groups
\begin{equation*}
0\to G\to H\to J\overset{\pi}{\to} 0.
\end{equation*}

We know that $i_{x*}$ is left-exact. Therefore, we will examine if $i_{x*}$ sends $\pi:H\to J$ to an epimorphism. Indeed
\begin{equation*}
i_{x*}(\pi)(U) = 
\begin{cases}
\pi \quad(x\in U),\\
*\quad(x\not\in U).
\end{cases}
\end{equation*}
Therefore, for each $U\subset X$ we have $\Coker(i_{x*}(\pi)(U))=0$. Thus $i_{x*}$ is exact.
\end{document}