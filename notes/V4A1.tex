\documentclass{article}

\usepackage{amsmath}
\usepackage{amssymb}
\usepackage{amsthm}
\usepackage{enumerate}
\usepackage{bbm}
\usepackage{lipsum}
\usepackage{fancyhdr}
\usepackage{calrsfs}
\usepackage{tikz-cd} 

\newtheorem{theorem}{Theorem}[section] 
\newtheorem{proposition}{Proposition}[section] 
\newtheorem{definition}{Definition}[section] 
\newtheorem{lemma}{Lemma}[section] 
\newtheorem{notation}{Notation}[section] 
\newtheorem{remark}{Remark}[section] 
\newtheorem{corollary}{Corollary}[section] 
\newtheorem{terminology}{Terminology}[section] 
\newtheorem{example}{Example}[section] 
\numberwithin{equation}{section}

\DeclareMathOperator{\diam}{diam}
\DeclareMathOperator{\rank}{rank}
\DeclareMathOperator{\Hom}{Hom}
\DeclareMathOperator{\Dom}{Dom}
\DeclareMathOperator{\grad}{grad}
\DeclareMathOperator{\Span}{Span}
\DeclareMathOperator{\interior}{int}
\DeclareMathOperator{\ind}{ind}
\DeclareMathOperator{\supp}{supp}
\DeclareMathOperator{\ob}{ob}
\DeclareMathOperator{\Spec}{Spec}
\DeclareMathOperator{\PreSh}{PreSh}
\DeclareMathOperator{\Sh}{Sh}
\DeclareMathOperator{\Fun}{Fun}
\DeclareMathOperator{\Ker}{Ker}
\DeclareMathOperator{\Image}{Im}
\DeclareMathOperator{\Coker}{Coker}

\title{Algebraic Geometry 1}
\author{So Murata}
\date{2024/2025 Winter Semester - Uni Bonn}

\begin{document}
\maketitle

\section{Topology}

\subsection{Connected Sets}

\begin{definition}
Let $(X,\mathcal{T})$ be a topological space. A subset $A$ of $X$ is said to be connected if for any $U,V\in\mathcal{T}$, $U\cap V=\empty, U\cup V\supset A$ then $A$ is fully contained in one of $U,V$. 
\end{definition}

\begin{definition}
A connected component of a topological space is a maximal connected subset of a space. 
\end{definition}

\begin{proposition}
Let $(X,\mathcal{T}_X),(Y,\mathcal{T}_Y)$ be topological space and $f:X\to Y$ be a continuous function. Then for any connected subset $A$ of $X$, $f(A)$ is connected in $Y$. 
\end{proposition}

\begin{proof}
\begin{align*}
U,V\in\mathcal{T}_Y, U\cup V\supset f(A), U\cap V=\emptyset, \\
 \Rightarrow &f^{-1}(U),f^{-1}(V)\in\mathcal{T}_X,\\
& f^{-1}(U)\cup f^{-1}(V)\supset A, \\
&f^{-1}(U)\cap f^{-1}(V)=\emptyset,\\
 \Rightarrow& f^{-1}(U)\supset A\lor f^{-1}(V)\supset A,\\
 \Rightarrow& U\supset f(A) \lor V\supset f(A).
\end{align*}
\end{proof}


\section{Category Theory}

\subsection{Categories}

\begin{definition}
A category $\mathcal{A}$ consists of
\begin{enumerate}[\textbullet]
\item a collection $\ob(\mathcal{A})$ of objects;
\item for each $A,B\in\ob(\mathcal{A})$, a collection $\mathcal{A}(A,B)$ of morphisms from $A$ to $B$;
\end{enumerate}
such that 
\begin{enumerate}[i).]
\item for each $A\in\ob(\mathcal{A})$, the identity $1_A\in\mathcal{A}(A,A)$;
\item the composition $\mathcal{A}(B,C)\times\mathcal{A}(A,B)\ni(g,f)\mapsto g\circ f\in\mathcal{A}(A,C)$ is well-defined;
\end{enumerate}
and they satisfy the following axioms
\begin{enumerate}[I).]
\item Associativity : $f\in\mathcal{A}(A,B),g\in\mathcal{A}(B,C),h\in\mathcal{A}(C,D)$, $(h\circ g)\circ f = h\circ (g\circ f)$.
\item Identity laws : $f\in\mathcal{A}(A,B)$ then $f\circ 1_A = 1_B\circ f$.
\end{enumerate}
\end{definition}

\begin{definition}
Let $\mathcal{A}$ be a category. A terminal object $T\in\ob(\mathcal{A})$ is an object such that for any $A\in\ob(\mathcal{A})$, $\mathcal{A}(A,T)$ is a single element set.
\end{definition}

\begin{definition}
Given two categories $\mathcal{A},\mathcal{B}$, we say $\mathcal{A}$ is a full-subcategory of $\mathcal{B}$ if 
\begin{enumerate}[i).]
\item $\mathcal{A}\subset\mathcal{B}$,
\item $\ob(\mathcal{A})=\ob(\mathcal{B})$.
\end{enumerate}
\end{definition}

\begin{notation}
Here we give notations to some important categories.
\begin{enumerate}[\textbullet]
\item $(\mathbf{Sets})$ : A category of sets equipped with set theoretic functions.
\item $(\mathbf{Ab})$ : A category of abelian groups with group homomorphisms.
\end{enumerate}
\end{notation}

\begin{example}
\label{order_category}
Given a partially ordered set $(X,\leq)$. This can be encoded to a category $\mathcal{O}$ by
\begin{enumerate}[i).]
\item $\ob(\mathcal{O}) = X$,
\item For $x,y\in X$, $x\leq y\Rightarrow\mathcal{O}(x,y) = \{*\}$ otherwise the morphisms between $x,y$ is an emptyset.
\end{enumerate}
\end{example}

\begin{definition}
A opposite/dual category of a category $\mathcal{A}$ is $\mathcal{A}^{\textbf{op}}$ such that
\begin{enumerate}[i).]
\item $\ob(\mathcal{A}^{\textbf{op}})=\ob(\mathcal{A})$,
\item $\mathcal{A}^{\textbf{op}}(B,A) = \mathcal{A}(A,B)$.
\end{enumerate}
\end{definition}

\begin{definition}
Let $\mathcal{A}$ be a category and $\varphi_1,\varphi_2\in\mathcal{A}(M,N)$. A morphism $\varphi:K\to M$ is called an equalizer of $(\varphi_1,\varphi_2)$ if for any morphism $\psi:P\to M$ such that
$\varphi_1\circ\psi=\varphi_2\circ\psi$, there is a unique morphism $\tilde{\psi}:P\to K$ such that $\varphi\circ\tilde{\psi}=\psi$.
\end{definition}

\begin{proposition}
If an equalizer exists then it is unique up to unique isomorphism.
\end{proposition}

\begin{proof}
Suppose $\varphi:K\to M,\psi:L\to M$ be equalizers of $(\varphi_1,\varphi_2)$. Then we have
\begin{align*}
\varphi\circ\tilde{\psi}=\psi,\quad\psi\circ\tilde{\varphi}=\varphi
\end{align*}
By the uniqueness, we have $\tilde{\varphi}\circ\tilde{\psi} = 1_L,\tilde{\psi}\circ\tilde{\varphi} = 1_K$. 
\end{proof}

\begin{definition}
Let $\mathcal{A},\mathcal{B}$ be categories. A functor $F:\mathcal{A}\to\mathcal{B}$ is a function such that for each $f\in\mathcal{A}(A,A')$, $F(f):F(A)\to F(A')$. In other words, $f\mapsto F(f):\mathcal{A}(A,A')\to\mathcal{B}(F(A),F(A'))$. Furthermore, $F$ satisfies the following axioms.
\begin{enumerate}[I).]
\item $F(f'\circ f) = F(f')\circ F(f)$ whenever $f:A\to A',f':A'\to A''$ in $\mathcal{A}$,
\item $F(1_A) = 1_{F(A)}$ whenever $A\in\mathcal{A}$. 
\end{enumerate}
\end{definition}

\begin{definition}
Let $F,G$ be functors between two categories $\mathcal{A},\mathcal{B}$. A natural transformation $\alpha:F\to G$ is a family $(\alpha_A:F(A)\to G(A))_{A\in\mathcal{A}}$ such that 
\[ \begin{tikzcd}
F(A) \arrow{r}{F(f)} \arrow[swap]{d}{\alpha_A} & F(A') \arrow{d}{\alpha_{A'}} \\%
G(A) \arrow{r}{G(f)}& G(A')
\end{tikzcd}
\]
is a commutative diagram. Each $\alpha_A$ is called a component of $\alpha$. 
 \label{natural_transformation}
\end{definition}

\subsection{Direct Limits}

\begin{definition}
A partially ordered set $(X,\leq)$ is directed if for any $x,y\in X$ there is $z\in X$ such that $x\leq c$ and $y\leq c$.
\end{definition}

\begin{example}
Let $(X,\mathcal{T})$ be a topological space. A partially ordered set $(\mathcal{T},\leq)$ such that 
\begin{equation*}
V\subseteq U\Rightarrow U\leq V
\end{equation*}
is directed. Since for any $U\in\mathcal{T}$, $U\leq \emptyset$. As a category this is $\mathbf{Ouv}_X^{\mathbf{op}}$.
\end{example}

\begin{example}
\label{top_dir_cat}
Let $(X,\mathcal{T})$ be a topological space. For $x\in X$, define $O_x = \{U\in\mathcal{T}\:|\: x\in U\}$. If we define an order as in the previous example, we get $(O_x,\leq)$ is directed. This follows from for any $U,V\in O_x$, $U,V\leq U\cap V$.  
\end{example}

\begin{definition}
Let $I$ be a directed partially ordered set and $\mathcal{A}$ be a category. A directed system of objects of $\mathcal{A}$ indexed by $I$ is a collection of objects $(A_i)_{i\in I}$ and morphisms $(\rho_{ij})_{i\leq j}$ of $\mathcal{A}$ such that 
\begin{enumerate}[i).]
\item $\rho_{ii}=\mathbf{id}_{A_i}$,
\item for $i,j,k\in I$, $i\leq j \leq k \Rightarrow \rho_{ik} = \rho_{jk}\circ\rho_{ij}$.
\end{enumerate}
\end{definition}

\begin{remark}
Categorically, the directed system of objects of $\mathcal{A}$ indexed by $I$ is a functor $\mathcal{O}^{op}\to\mathcal{C}$, where $\mathcal{O}$ is a category which encodes the ordered set $I$ as a category by the same procedure as in Example \ref{order_category}. Then a directed system if a functor $\mathcal{O}^{\mathbf{op}}\to\mathcal{A}$.
\end{remark}

\begin{definition}
Given a directed system $((A_i)_{i\in I},\{\rho_{ij}\}_{i\leq j})$ of objects in $\mathcal{A}$ indexed by $I$. A direct limit of the system is an object $\varinjlim{A_n}\in\mathbf{ob}(\mathcal{A})$ satisfying the following universal property.\\
\par Given a collection of morphisms $(f_i)_{i\in I}$ such that 
\begin{enumerate}[i).]
\item $f_i:A_i\to\varinjlim A_n\in\mathcal{A}$,
\item for any $i\leq j$, $f_j\circ\rho_{ij} = f_i$.
\end{enumerate}
For any $A\in\mathcal{A}$ where there is a collection of morphisms $(g_i)_{i\in I}$ satisfying the above condition, there is a unique map $\varphi:\varinjlim A_n\to A$ such that

\[ 
\begin{tikzcd}
A_i \arrow[rr, "\rho_{ij}", no head] \arrow[rd, "f_i"] \arrow[rdd, swap, "g_i", bend right] &                                               & A_j \arrow[ld, "f_j"'] \arrow[ldd, "g_j", bend left] \\
                                                                                      & \varinjlim A_n \arrow[d, "\exists !", dotted] &                                                      \\
                                                                                      & A                                             &                                                     
\end{tikzcd}
\]

is a commutative diagram.
\end{definition}

\begin{proposition}
In the cases where $\mathcal{A}=(\mathbf{Ab}),(\mathbf{Sets})$, there exist direct limits and for each category, such limit is constructed in the following ways.
\begin{enumerate}[i).]
\item $\varinjlim A_n = (\bigoplus_{i\in I}A_i)/N$ where $N=\{a_i-\rho_{ij}(a_i)\:|\:a_i,i\leq j\}$.
\item $\varinjlim A_n = (\coprod_{i\in I}A_i)/\sim$ where $a_i\sim a_j$ if there is $k$ such that $i\leq k$ $j\leq k$, and $\rho_{ik}(a_i)=\rho_{jk}(a_j)$.
\end{enumerate}
Furthermore, these two direct limits match as sets.
\end{proposition}

\begin{proposition}
$\varinjlim$ is (left) exact in $(\mathbf{Ab})$. In other words, 
given a exact sequence of directed systems 
\[
\begin{tikzcd}
0 \arrow[r] & (M_i)_{i\in I} \arrow[r] & (N_i)_{i\in I} \arrow[r] & (P_i)_{i\in I} \arrow[r] & 0
\end{tikzcd}
\]
in which we have
\[
\begin{tikzcd}
0 \arrow[r] & M_i \arrow[r] \arrow[d, "\rho^M_{ij}"'] & N_i \arrow[r] \arrow[d, "\rho^N_{ij}"'] & P_i \arrow[r] \arrow[d, "\rho^P_{ij}"'] & 0 \\
0 \arrow[r] & M_j \arrow[r]                           & N_j \arrow[r]                           & P_j \arrow[r]                           & 0
\end{tikzcd}
\]
There exists a short exact sequence 
\[
\begin{tikzcd}
0 \arrow[r] & \varinjlim M_n \arrow[r] & \varinjlim N_n \arrow[r] & \varinjlim P_n \arrow[r] & 0
\end{tikzcd}
\]
\end{proposition}

\section{Sheaf Theory}

\subsection{Presheaves}

\begin{definition}
Let $(X,\mathcal{T})$ be a topological space. We define the presheaf $\mathcal{F}$ of a category $\mathcal{A}$ on $X$ such that 
\begin{enumerate}[\textbullet]
\item $U\in\mathcal{T}$, $\mathcal{F}(U)\in\ob(\mathcal{A})$,
\item $U,V\in\mathcal{T}$, $V\subset U \Rightarrow $ there exists a map $\rho_{UV}:\mathcal{F}(U)\to\mathcal{F}(V)$ 
\end{enumerate}

such that 
\begin{enumerate}[i).]
\item For any $U\in\mathcal{T}$, $\rho_{UU}=1_{\mathcal{F}(U)}$.
\item $U,V,W\in\mathcal{T}, W\subset V\subset U\rightarrow \rho_{UW}=\rho_{VW}\circ\rho_{UW}$.
\end{enumerate}
\end{definition}

\begin{remark}
In the case $\mathcal{A} = (\mathbf{Sets}),(\mathbf{Ab})$, $\mathcal{F}(\emptyset)=\emptyset,\{1\}$, respectively.
\end{remark}

\begin{definition}
An element of $\mathcal{F}(U)$ is called a local section of $\mathcal{F}$ and $\Gamma(U,\mathcal{F}) = \mathcal{F}(U)$ is called the space of sections over $U$. In particular $\Gamma(X,\mathcal{F})$ is called the space of global sections of $\mathcal{F}$.
\end{definition}

\begin{definition}
Let $(X,\mathcal{T})$ be a topological space and $\mathcal{F}$ be a presheaf of a category $\mathcal{A}$ on $X$. Suppose we have two open sets $U,V\in\mathcal{T}$ such that $V\subset U$. Then for any section $s\in\mathcal{F}(U)$, $s|_V=\rho_{UV}(s)$ is called the restriction of $s$ to $V$.
\end{definition}

\begin{example}
Let $(X,\mathcal{T})$ be a topological space. We have a presheaf of continuous functions $\mathcal{C}_X(U)=\mathcal{C}^0(U,\mathbb{R})$. This is indeed a presheaf with restriction maps $\rho_{UV}:\mathcal{C}_X(U)\to\mathcal{C}_X(V)$. (Explicitly, $\rho_{UV}(f) = f\circ i_V$ where $i_V$ is an inclusion map.)  We note that we can introduce operations $+,\cdot$ to endow some algebraic structures (groups, rings, ...) on $\mathbb{R}$.
\end{example}

\begin{example}
Let $(X,\mathcal{T})$ be a topological space and suppose we have presheaves 
\begin{enumerate}[\textbullet]
\item $\mathcal{C}^{\textbf{diff}}_X(U) = \{f:U\to\mathbb{R}\:|\: f\text{ is differentiable.}\}$.
\end{enumerate}
Then there is an inclusion relation $\mathcal{C}^{\textbf{diff}}_X(U)\subseteq\mathcal{C}_X(U)$ and this defines a presheaf.
\end{example}

\begin{example}
Let $(X,\mathcal{T}_X)$,$(Y,\mathcal{T}_Y)$ be topological spaces. Define a presheaf on $X$ by 
\begin{equation*}
U\in\mathcal{T}_X, \mathcal{F}(U) = \mathcal{C}^0(X,Y).
\end{equation*}
And like the previous example, we define $\rho_{UV}(f) = f|_V$ for $U,V\in\mathcal{T}_X, V\subset U$. the restriction of $f$ to $V$. \\
But this is a presheaf only of a set. 
\end{example}

\begin{example}
Let $(X,\mathcal{T})$ be a topological space and $G$ be an abelian group. The constant presheaf $\mathbb{G}$ is such that 
\begin{equation*}
U\in\mathcal{T}, \mathbb{G}(U) = G,
\end{equation*}
with $\rho_UV=id_G$ for any $U,V\in\mathcal{T},V\subset U$. 
\end{example}

\subsection{Presheaves as Categories}

\begin{definition}
Let $(X,\mathcal{T})$ be a topological space then $(\mathbf{Ouv}_X)$ is the category such that its objects are the open sets of $X$ and for any $U,V\in\mathcal{T}$ we have
\begin{equation*}
\mathbf{Ouv}_X(U,V)=
\begin{cases}
\emptyset \quad (V\not\subset U),\\
i_V \quad (V\subset U).
\end{cases}
\end{equation*}
\end{definition}

\begin{definition}
Let $(X,\mathcal{T})$ be a topological space and $\mathcal{A}$ be a category. A presheaf of $\mathcal{A}$ on $X$ is a functor $F:\mathbf{Ouv}_X\to\mathcal{A}$.
\end{definition}

\begin{example}
For $\mathbf{Ouv}_X$, we can define a presheaf of $F$ to be
\begin{equation*}
\ob(\mathbf{Ouv}_X)\ni U\mapsto F(U) = \mathcal{C}^0(U,\mathbb{R}).
\end{equation*}
\end{example}

\begin{example}
\label{structure_sheaf_ring}
Let $A$ be a commutative ring with non-zero multiplicative identity and $X=\Spec(A)$. Let us consider the Zariski topology $(X,\mathcal{T})$. Let us consider a category $\mathcal{O}_X$ such that
\begin{enumerate}[\textbullet]
\item $\ob(\mathcal{O}_X) = \mathcal{T}$,
\item $\mathcal{O}_X(U) =  \{s:U\to\coprod_{\mathfrak{p}\in U}A_{\mathfrak{p}}\}$,
\end{enumerate}
where $s:U\to\coprod_{\mathfrak{p}\in U}A_{\mathfrak{p}}$ is a function such that for any $\mathfrak{p}\in U$,
\begin{enumerate}[i).]
\item $s(p)\in A_{\mathfrak{p}}$,
\item there exists an open set $V\subset U$ such that $\mathfrak{p}\in V$ and for any $\mathfrak{q}\in V$,  $s(\mathfrak{q})={\frac a b}$ for $b\not\in\mathfrak{q}$. 
\end{enumerate}
Now we define a presheaf by the restrictions of maps such that
\begin{equation*}
s:U\to\coprod_{\mathfrak{p}\in U}A_{\mathfrak{p}} \mapsto s|_V:V\to\coprod_{\mathfrak{q}\in V} A_{\mathfrak{q}}.
\end{equation*}
\end{example}

\begin{definition}
Let $(X,\mathcal{T})$ be a topological space and $\mathcal{A}$ be a category. We define a set of presheaves of $\mathcal{A}$ on $X$ as
\begin{equation*}
\PreSh_{\mathcal{A}}(X)=\Fun(\mathbf{Ouv}_X^{\mathbf{op}},\mathcal{A}).
\end{equation*}
\end{definition}

\begin{definition}
A morphism of presheaves is a natural transformation $\varphi:\mathcal{F}\to \mathcal{G}$ where $\mathcal{F},\mathcal{G}\in\PreSh_\mathcal{A}(X)=\Fun(\mathbf{Ouv}_X^{\mathbf{op}},\mathcal{A})$.(See Definition \ref{natural_transformation}).\\
\par Such $\varphi:\mathcal{F}\to\mathcal{G}$ is 
\begin{enumerate}[i).]
\item injective if 
\end{enumerate}
\end{definition}



\begin{remark}
$\PreSh(X)$ can be regarded as a category with its objects presheaves and morphisms defined above. 
\end{remark}

\begin{notation}
In the case $\mathcal{A}= (\mathbf{Ab})$ then we denote $\PreSh(X)=\PreSh_{\mathbf{Ab}}(X)$.
\end{notation}

\begin{example}
Let $X$ be a differential manifold(eg. $X\subset\mathbb{R}^n$). Let us define 
\begin{equation*}
\mathcal{C}^{\mathbf{diff}}(U) = \{f:U\to\mathbb{R}\:|\:f\text{ is differentiable.}\}. 
\end{equation*}
Then the inclusions $\mathcal{C}_X^{\mathbf{diff}}(U)\subset\mathcal{C}_X(U)$ defines the natural transformation.
\end{example}

\begin{example}
Let $X,Y=S^1$ be topological spaces and $F$ be a presheaf such that for any open set $U\subset X$, $F(U)=\mathcal{C}^0(U,Y)$.Then we can introduce a natural transformation such that 
\begin{equation*}
\mathcal{C}_X(U)\ni f\mapsto \exp(2\pi fi).
\end{equation*}
\end{example}

\subsection{Sheaves}

\begin{definition}
A presheaf $\mathcal{F}$ on $(X,\mathcal{T})$ is called a sheaf if the following holds.
For any collection of open sets $(U_i)_{i\in I}\subset \mathcal{T},U=\bigcup_{i\in I}U_i$, the map $\varphi:\mathcal{F}(U)\to\prod_{i\in I}\mathcal{F}(U_i)$ which is defined as
\begin{equation*}
\varphi(s) = (s|_{U_i})_{i\in I}.
\end{equation*}
is the equalizer of the following functions $\varphi_1,\varphi_2:\prod_{i\in I}\mathcal{F}(U_i)\to\prod_{i,j\in I}\mathcal{F}(U_i\cap U_j)$, 
\begin{equation*}
\varphi_1((s_i)_{i\in I}) = (s_i|_{U_i\cap U_j})_{i,j\in I},\quad\varphi_1((s_i)_{i\in I}) = (s_j|_{U_i\cap U_j})_{i,j\in I}.
\end{equation*}
\end{definition}

\begin{remark}
In the case $I=\{1,2\}$, we have $U=U_1\cup U_2$, and for any $U'\in\mathcal{T}$ such that $U\subset U'$, we have for $\mathcal{F}(U')\ni s:U'\to\mathbb{R}$ ,$\psi(s) = (s|_{U_1},s|_{U_2})$, as in $\mathbf{Ouv}_X$, morphisms are inclusions. Let $\tilde{\psi}(s) = s|_{U}$, then this satisfies the condition for the equalizer (ie. $\varphi\circ\tilde{\psi}=\psi$). 
\end{remark}

\begin{remark}
A presheaf $\mathcal{O}_X$ with $X=\Spec(A)$ is a sheaf.
\end{remark}

\begin{example}
Let $(X,\mathcal{T})$ be a topological space and $G$ be a group. We define a constant presheaf $\mathbb{G}(U) = G$. In general, this is not a sheaf. Instead, we define a constant sheaf $\underline{\mathbb{G}}(U) = \mathcal{C}^0(U,G)$ where $G$ is regarded as a topological space with the discrete topology. Then for any connected component of $X$ is mapped to a single point set in $G$.
\end{example}

\begin{definition}
Let $\mathcal{F}_1,\mathcal{F}_2$ be sheaves. A mapping $\varphi:\mathcal{F}_1\to\mathcal{F}_2$ is called a morphism of sheaves if it is a morphism of presheaves.
\end{definition}

\begin{definition}
A set of sheaves of $\mathcal{A}$ on the topological space $(X,\mathcal{T})$ is denoted as $\Sh_{\mathcal{A}}(X)$.
\end{definition}

\begin{remark}
As in the case of presheaves, $\Sh_\mathcal{A}(X)$ can be regarded as a category with sheaf morphisms.
\end{remark}

\begin{remark}
$\Sh_{\mathcal{A}}(X)$ is a full-subcategory of $\PreSh_{\mathcal{A}}(X)$.
\end{remark}

\begin{notation}
In the case $\mathcal{A}=(\mathbf{Ab})$, we denote $\Sh_{(\mathbf{Ab})}(X) = \Sh(X)$. 
\end{notation}

\subsection{Stalks}

\begin{definition}
\label{def_stalk}
Suppose we have a topological space $(X,\mathcal{T})$ and a category $\mathcal{A}$ which admits direct limits. For a presheaf $\mathcal{F}\in\PreSh_{\mathcal{A}}(X)$, by inheriting the notations from Example \ref{top_dir_cat}, we define the stalk $\mathcal{F}_x$ of $\mathcal{F}$ at $x\in X$ by
\begin{equation*}
\mathcal{F}_x = \varinjlim_{U\in\mathcal{O}_x}\mathcal{F}(U) =  \varinjlim_{x\in U,U\in\mathcal{T}}\mathcal{F}(U).
\end{equation*}
\end{definition}

\begin{example}
\label{stalk_ex1}
Let us assume that $\mathcal{A}=(\mathbf{Ab})$ in Definition \ref{def_stalk}. Then stalks and germs can be constructed explicitly in the following way.
\begin{equation*}
\mathcal{F}_x = \{(s,U)\:|\: U\in\mathcal{O}_x, s\in\mathcal{F}(U)\}/\sim,
\end{equation*}
where $\sim$ is an equivalent relation such that for $(s,U),(t,V)$,
\begin{equation*}
(s,U)\sim(t,V) \text{ if there is } W\in\mathcal{O}_x \text{ such that } W\subseteq U\cap V, \rho_{UW}(s)=\rho_{VW}(t).
\end{equation*}
\end{example}

\begin{definition}
Inheriting the notations from Definition \ref{def_stalk}, suppose we have $(f_U:\mathcal{F}(U)\to\mathcal{F}_x)_{U\in\mathcal{O}_x}$ such that for $f_U, f_V$ are compatible with $\rho_{UV}$. Then we define the germ of $s\in\mathcal{F}(U)$ to be $s_x=f_U(s)$. By the universal property of the direct limit, such $s_x$ is unique up to images under isomorphisms.
\end{definition}

\begin{example}
In the case of Remark \ref{stalk_ex1}, we have for each $U\in\mathcal{T}$, $x\in U$, and $s\in\mathcal{F}(U)$,
\begin{equation*}
s_x = \{(t,V)\:|\: \text{ There is } W\in\mathcal{O}_x \text{ such that } W\subseteq U\cap V, \rho_{UW}(s)=\rho_{VW}(t)\}.
\end{equation*}
\end{example}

\begin{remark}
In the above definition, if a category $\mathcal{A}$ admits products, we get a map
\begin{equation}
\label{germ_map}
(s\mapsto (s_x)_{x\in U})):\mathcal{F}(U)\to\prod_{x\in U}\mathcal{F}_x.
\end{equation}
This is neither surjective nor injective in general.
\end{remark}

\begin{proposition}
\label{stalk_lifting}
Suppose in the definition of stalks, $\mathcal{F}$ is a sheaf. Then the map defined by Equation \ref{germ_map} is injective.
\end{proposition}

\begin{proof}
We prove the case when $\mathcal{A}=(\mathbf{Ab})$. \\
\par Suppose $s\in\mathcal{F}(U)$ is such that $s_x=0$ in $\mathcal{F}_x$ for all $x\in U$. Since for any restriction maps are group homomorphisms. We have that there is $V_x\in\mathcal{O}_x$ such that 
\begin{equation*}
V_x\subseteq U,\quad\rho_{UV_x}(s) = 0.
\end{equation*}
Therefore $\{V_x\}_{x\in U}$ is an open covering of $U$. Since $\mathcal{F}$ is a sheaf, we derive that $s=0$ in $\mathcal{F}(U)$.
\end{proof}

\begin{example}
Given $(X,\mathcal{F})$, a topological space and $G$, an abelian group. We will consider the constant presheaf $\mathbb{G}$ and the constant sheaf $\underline{\mathbb{G}}$ on $X$. For any open set $U$ and $x\in U$ we have
\begin{equation*}
\mathbb{G}_x\cong\underline{\mathbb{G}}_x\cong G.
\end{equation*}
For any $U,V$ open such that $V\subset U$ we have, $\rho_{UV}=\mathbf{id}_G$. Thus by the construction, for $x\in U,V$, $(s,U)\sim (t,V)$ then $x\in U\cap V$ and $\rho_{UU\cap V}(s) = s=t=\rho_{VU\cap V}(t)$. Therefore, we proved the claim.
\end{example}

\begin{remark}
Categorically, taking stalks is a functor for each $x\in X$. Suppose we have $\mathcal{F},\mathcal{G}\in\PreSh_\mathcal{A}(X)$ and a morphism $\varphi:\mathcal{F}\to\mathcal{G}$, then 
later
\end{remark}

\begin{proposition}
Let $\mathcal{F},\mathcal{G}\in\Sh_{(\mathbf{Ab})}(X)$ Then for any morphism $\varphi:\mathcal{F}\to\mathcal{G}$ we have
\begin{equation*}
\varphi = 0 \Leftrightarrow \forall x\in X,  \varphi_x = 0
\end{equation*}
\end{proposition}

\begin{proof}
$\Rightarrow$ is trivial by its construction. We will prove $\Leftarrow$.\\
\par We first note that $\varphi=0$ means that for any $U\in \mathcal{T}$, we have $\varphi_U\equiv0$ as a group homomorphism. Let $U\in\mathcal{T}$ and $s\in\mathcal{F}(U)$. Then by the assumption and Proposition \ref{stalk_lifting}, we have proven the claim.
\end{proof}

\subsection{Sheafification}

\begin{definition}
Let $\mathcal{F}\in\PreSh_\mathcal{A}(X)$. The sheafification of $\mathcal{F}$ is a presheaf $\mathcal{F}^+$ which is a set of all $(s_x)_{x\in U}\in\prod_{x\in U}\mathcal{F}_x$ such that
for any $x\in U$ there is $x\in V_x\subset U$, such that there is $t\in\mathcal{F}(V_x)$ satisfying for any $y\in V_x$, $s_y = t_y$. We give them restrictions such that
\begin{equation*}
\mathcal{F}^+(U)\ni (s_x)_{x\in U}\mapsto (s_x)_{x\in V}\in \mathcal{F}^+(V).
\end{equation*}
\end{definition}

\begin{proposition}
Such $\mathcal{F}^+$ is indeed a sheaf.
\end{proposition}

\begin{proof}
later
\end{proof}

\begin{remark}
\begin{equation*}
\mathcal{F}\mapsto\mathcal{F}^+:\PreSh_\mathcal{A}(X)\to\Sh_\mathcal{A}(X)
\end{equation*}
is a functor. Indeed given $\varphi:\mathcal{F}\to\mathcal{G}$, a morphism of presheaves. We give
\begin{equation*}
\varphi^+(U)((s_x)_{x\in U}) = (\varphi(s)_x)_{x\in U}.
\end{equation*}
later
\end{remark}

\begin{proposition}
A mapping $\varphi:\mathcal{F}\to\mathcal{F}^+$ such that for each $U\in\mathcal{T}$, 
\begin{equation*}
\varphi_U:\mathcal{F}(U)\to\mathcal{F}^+(U),\quad \varphi(s)=(s_x)_{x\in U},
\end{equation*}
is a natural transformation thus a morphism of presheaves.
\end{proposition}

\begin{proof}
Later
\end{proof}

\begin{proposition}
\label{section_rep}
For any open set $U\in\mathcal{T}$ and a section $s\in\mathcal{F}^+(U)$, there is an open covering $(U_i)_{i\in I}$ which satisfies that there is a sequence $(s_i)_{i\in I}\in\prod_{i\in I}\mathcal{F}(U_i)$ and for each $i$, the following hodls.
\begin{equation*}
\rho_{UU_i}(s)=s_i.
\end{equation*}
\end{proposition}
\begin{proof}
Later.
\end{proof}

\begin{proposition}
For each $x\in X$, there exists an isomorphism
\begin{equation*}
\mathcal{F}_x\cong (\mathcal{F}^+)_x,
\end{equation*}
as presheaves.
\end{proposition}

\begin{proof}
later
\end{proof}

\begin{proposition}
Let $(X,\mathcal{T})$ be a topological group and $\mathcal{F}$ be a presheaf of a category $\mathcal{A}$ on $X$. Suppose for a sheaf $\mathcal{G}$ of a category $\mathcal{A}$ on $X$, there exists a morphism $\varphi:\mathcal{F}\to\mathcal{G}$. Then there exists a unique morphism $\varphi^+:\mathcal{F}^+\to\mathcal{G}$, such that
\[
\begin{tikzcd}
\mathcal{F} \arrow[d, "\varphi"'] \arrow[r] & \mathcal{F}^+ \arrow[ld, "\exists!\varphi^+", dotted] \\
\mathcal{G}                                 &                                                      
\end{tikzcd}
\]
is a commutative diagram.
\label{sheaf_morphism_universal_property}
\end{proposition}

\begin{proof}
Let $U\in\mathcal{T}$, then by Proposition \ref{section_rep}, for any $s\in\mathcal{F}^+$, there exists an open covering $(U_i)_{i\in I}$ and $(s_i)_{i\in I}\in\prod_{i\in I}\mathcal{F}(U_i)$ such that $\rho_{UU_i}(s)=s_i$ for any $i\in I$. We define
\begin{equation*}
t_i = \varphi(s_i)\in\mathcal{G}(U_i),
\end{equation*}
for each $i\in I$. Using the definition of natural transformation we derive that
\begin{equation*}
\rho^{\mathcal{G}}_{UU_i\cap U_j}(t_i) = \varphi^{\mathcal{F}}_{U_i\cap U_j}(\rho_{UU_i\cap U_j}(s))=\rho^{\mathcal{G}}_{UU_i\cap U_j}(t_j).
\end{equation*}
Thus we can glue $(t_i)_{i\in I}$ to a section $t\in\mathcal{G}(U)$.\\
\par We now define $\varphi^+:\mathcal{F}^+\to\mathcal{G}$. Given $(s_x)_{x\in U}$ which is the germ of $s$,
\begin{equation*}
\varphi^+_{U}((s_x)_{x\in U}) = t.
\end{equation*}
Such $\varphi^+$ is unique since $\mathcal{G}$ is a sheaf.
\end{proof}

\begin{corollary}
Let $i:\Sh_\mathcal{A}(X)\to\PreSh_\mathcal{A}(X)$ be a forgetful functor. Then we have
\begin{equation*}
\PreSh_\mathcal{A}(X)(\mathcal{F},i(\mathcal{G})) \cong \Sh_{\mathcal{A}}(\mathcal{F}^+,\mathcal{G})
\end{equation*}
In other words, the sheafification is a left-adjoint functor of the inclusion map.
\end{corollary}

\begin{proof}
By Proposition \ref{sheaf_morphism_universal_property}, we define two maps $\Phi,\Psi$ such that
\begin{align*}
\Phi:\PreSh_\mathcal{A}(X)(\mathcal{F},i(\mathcal{G})) &\to \Sh_{\mathcal{A}}(\mathcal{F}^+,\mathcal{G}),&\\
& \Phi(\varphi) = \varphi^+,\\
\Psi:\Sh_{\mathcal{A}}(\mathcal{F}^+,\mathcal{G})&\to\PreSh_\mathcal{A}(X)(\mathcal{F},i(\mathcal{G})),&\\
 &\Psi(\varphi^+)  = \varphi.
\end{align*}
Then these two are inverses of each other.
\end{proof}

\begin{proposition}
Let $X=\Spec(A)$ and $\mathcal{O}_X$ be the structure sheaf defined in Example \ref{structure_sheaf_ring}. Then we have the following.
\begin{enumerate}[1).]
\item For any $\mathfrak{p}=x\in X$, $(\mathcal{O}_{X})_x \cong A_{\mathfrak{p}}$.
\item For any $a\in A$, $\mathcal{O}_X(D(a)) \cong A_a$.
\end{enumerate}
\end{proposition}

\begin{proof}
For a given $U\subset X$ open and $\mathfrak{p}\subset A$, there is $a,b\in A$ such that for $V\subset U$ open and $s\in\mathcal{O}_X(U), s:U\to\coprod_{\mathfrak{p}\in U}A_{\mathfrak{p}}$.
\begin{equation*}
s(\mathfrak{q}) = {\frac a b}\in A_{\mathfrak{q}}
\end{equation*}
holds for any $\mathfrak{q}\in V$.
\[
\begin{tikzcd}
\mathcal{O}_X(U) \arrow[r] \arrow[d, "\rho_{UV}"'] & A_\mathfrak{p} \\
\mathcal{O}_X(V) \arrow[ru]                        &               
\end{tikzcd}
\]
\end{proof}

\begin{definition}
Let $\varphi:\mathcal{F}\to\mathcal{G}$ be a homomorphism of presheaves $\PreSh_{(\mathbf{Ab})}(X)$. Then we define the following.
\begin{enumerate}[1).]
\item $\Ker^{\mathbf{pre}}(\varphi)(U) = \Ker\varphi_U$,
\item $\Image^{\mathbf{pre}}(\varphi)(U) = \Image\varphi_U$,
\item $\Coker^{\mathbf{pre}}(\varphi)(U) = \Coker\varphi_U$.
\end{enumerate}
\end{definition}

\begin{proposition}
Such $\Ker^{\mathbf{pre}},\Image^{\mathbf{pre}},\Coker^{\mathbf{pre}}$ are presheaves.
\end{proposition}

\begin{proof}For the case of kernels.
Let $U,V\in\mathcal{T}$ and $V\subset U$. We define $\rho_UV: \Ker^{\mathbf{pre}}(\varphi)(U) \to\Ker^{\mathbf{pre}}(\varphi)(V)$ to be such that
\begin{equation*}
\rho_UV(s) = \rho^{\mathcal{F}}(s).
\end{equation*}
Such construction is justified as the diagram below is commutative.
\[
\begin{tikzcd}
\mathcal{F}(U) \arrow[r, "\rho^{\mathcal{F}}_{UV}"] \arrow[d, "\varphi_U"'] & \mathcal{F}(V) \arrow[d, "\varphi_V"] \arrow[r, "\rho^{\mathcal{F}}_{UV}"] & \mathcal{F}(W) \arrow[d, "\varphi_W"] \\
\mathcal{G}(U) \arrow[r, "\rho^{\mathcal{G}}_{UV}"']                        & \mathcal{G}(V) \arrow[r, "\rho^{\mathcal{G}}_{UV}"']                       & \mathcal{F}(W)                       
\end{tikzcd}
\]
Furthermore,
\begin{equation*}
\rho_UW(s) = \rho^{\mathcal{F}}_{UV}(s)=\rho^{\mathcal{F}}_{VW}\circ\rho^{\mathcal{F}}_{UV}(s) = \rho_{VW}\circ\rho_{UV}(s).
\end{equation*}
Thus $\Ker^{\mathbf{pre}}(\varphi)(U)$ is a presheaf.
\end{proof}

\begin{corollary}
If $\varphi:\mathcal{F}\to\mathcal{G}$ is a morphism of sheaves. Then $\Ker^{\mathbf{pre}}$ is also a sheaf.
\end{corollary}

\begin{proof}
Given $(s_i)_{i\in I}\in\prod_{i\in I}\Ker\varphi_{U_i}$ such that
\begin{equation*}
\rho(s_i)_{U_iU_i\cap U_j}=\rho(s_j)_{U_jU_i\cap U_j}
\end{equation*}
for any $i,j\in I$. Then since $\mathcal{F}$ is a sheaf, we can glue $(s_i)_{i\in I}$ to $s\in\mathcal{F}(U)$. For such $s$ we have
\begin{equation*}
\rho^{\mathcal{G}}_{UU_i}(\varphi_U(s)) = (\varphi_{U_i}(\rho^{\mathcal{F}}_{UU_i}(s))) = \varphi_{UU_i}(s_i) = 0.
\end{equation*}
Therefore, since $\mathcal{G}$ is a sheaf, $\varphi_U(s)=0$.
\end{proof}

\begin{remark}
Let $\varphi:\mathcal{F}(U)\to\prod_{i\in I}\mathcal{F}(U_i),\varphi_1:\prod_{i\in I}\mathcal{F}(U_i)\to\prod_{i,j\in I}\mathcal{F}(U_i\cap U_j),\varphi_2:\prod_{i\in I}\mathcal{F}(U_j)\to\prod_{i,j\in I}\mathcal{F}(U_i\cap U_j)$. Then $\mathcal{F}$ is a sheaf if and only if
\begin{equation*}
\Ker(\varphi_1\circ\varphi - \varphi_2\circ\varphi) = \mathcal{F}(U),
\end{equation*}
holds for any open set $U$.
\end{remark}

\begin{remark}
$\Image^{\mathbf{pre}}\varphi,\Coker^{\mathbf{pre}}\varphi$ are not in general sheaves even tho $\varphi:\mathcal{F}\to\mathcal{G}$ is a homomorphism of sheaves.
\label{sheaf_condition_kernel}
\end{remark}

\begin{example}
Let $X=\{x_1,x_2\}$ and we assign the discrete topology to it. Let $G$ be an abelian group. We define a sheaf $\mathcal{F},\mathcal{G}\in\Sh_{(\mathbf{Ab})}(X)$ by such that
\begin{equation*}
\mathcal{F}(U)=\mathcal{G}(U) = 
\begin{cases}
G\times G\quad (U=X),\\
G \quad (|U|=1),\\
0\quad (U=\emptyset).
\end{cases}
\end{equation*}
Let us define a homomorphism of sheaves $\varphi$ such that
\begin{equation*}
\varphi_U = 
\begin{cases}
\mathbf{id}_{G\times G}\quad (U=X)\\
0\quad (U\not= X).
\end{cases}
\end{equation*}
Then we have
\begin{equation*}
\Coker^{\mathbf{pre}}(\varphi)(U) = 
\begin{cases}
0\quad (U=X),\\
G\quad (U\not= X).
\end{cases}
\end{equation*}
By \ref{sheaf_condition_kernel}, we observe that
\begin{equation*}
\Coker^{\mathbf{pre}}(\varphi)(X)=G\times G/\mathbf{id}_{G\times G}(G\times G) = \{0\}.
\end{equation*}
However, 
\begin{equation*}
later.
\end{equation*}
\end{example}

\begin{definition}
Given a morphism of sheaves $\varphi:\mathcal{F}\to\mathcal{G}$, we define the following.
\begin{enumerate}[1).]
\item $\Ker(\varphi)=\Ker^{\mathbf{pre}}(\varphi)$,
\item $\Image(\varphi)=(\Image^{\mathbf{pre}}(\varphi))^+$,
\item $\Coker(\varphi)=(\Coker^{\mathbf{pre}}(\varphi))^+$.
\end{enumerate}
\end{definition}

\begin{proposition}[Universal property of kernels]
Given a sheaf homomorphism $\varphi:\mathcal{F}\to\mathcal{G}$. For any sheaf homomorphism $\alpha:\mathcal{H}\to\mathcal{F}$, $\varphi\circ\alpha = 0$ if and only if there is a unique $\psi:\mathcal{H}\to\Ker\varphi$ such that
\[
\begin{tikzcd}
                                        & \mathcal{H} \arrow[d, "\alpha"] \arrow[rd, "\varphi_0=0"] \arrow[ld, "\exists!\psi"', dotted] &             \\
\Ker(\mathcal{\varphi}) \arrow[r, hook] & \mathcal{F} \arrow[r, "\varphi"']                                                           & \mathcal{G}
\end{tikzcd}
\]
is a commutative diagram.
\end{proposition}
\begin{proof}
We argue by each open set of the space.
\[
\begin{tikzcd}
                                        & \mathcal{H}(U) \arrow[d, "\alpha_U"] \arrow[rd, "(\varphi_0)_U=0"] \arrow[ld, "\exists!\psi_U"', dotted] &             \\
\Ker(\mathcal{\varphi})(U) \arrow[r, hook] & \mathcal{F}(U) \arrow[r, "\varphi_U"']                                                           & \mathcal{G}(U)
\end{tikzcd}
\]
This is a universal property of the kernel in abelian groups. Thus the statement immediately follows from it.
\end{proof}

\begin{proposition}[Universal property of Cokernels]
Given a sheaf homomorphism $\varphi:\mathcal{F}\to\mathcal{G}$. For any sheaf homomorphism $\alpha:\mathcal{G}\to\mathcal{H}$, $\alpha\circ\varphi = 0$ if and only if there is a unique $\psi:\Coker\varphi\to\mathcal{H}$ such that
\[
\begin{tikzcd}
\mathcal{F} \arrow[r, "\varphi"] \arrow[r] \arrow[rd, "\varphi_0=0"'] & \mathcal{G} \arrow[r,"\pi"] \arrow[d, "\alpha"]                 & \Coker(\varphi) \arrow[ld, "\exists!\psi"', dotted, swap]\\
                                                                    & \mathcal{H}  &                
\end{tikzcd}
\]
is a commutative diagram.
\end{proposition}

\begin{proof}
We argue for each open set $U\subset X$. 
\[
\begin{tikzcd}
\mathcal{F}(U) \arrow[r, "\varphi_U"] \arrow[r] \arrow[rd, "(\varphi_0)_U=0"'] & \mathcal{G}(U) \arrow[r] \arrow[d,"\alpha_U"] & \Coker^{\mathbf{pre}}(\varphi)(U) \arrow[ld, "\exists!\psi^{\mathbf{pre}}_U"', dotted] \arrow[r]& \Coker(\varphi)(U) \arrow[lld, "\exists!\psi_U", dotted] \\
                                                                    & \mathcal{H}(U)                     &                                                                                   &                                                    
\end{tikzcd}
\]
By the universal property of Cokernels of abelian groups, there is a unique $\varphi^{\mathbf{pre}}$. By the universal property of the sheafification operator, we derive a unique $\psi$.
\end{proof}

\begin{proposition}
Let $x\in X$, then we have the following.
\begin{enumerate}[1).]
\item $\Ker(\varphi)_x=\Ker(\varphi_x)$,
\item $\Image(\varphi)_x=\Image(\varphi_x)$,
\item $\Coker(\varphi)_x=\Coker(\varphi_x)$.
\end{enumerate}
\end{proposition}

\end{document}